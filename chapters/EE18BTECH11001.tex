\begin{enumerate}[label=\thesection.\arabic*.,ref=\thesection.\theenumi]
\numberwithin{equation}{enumi}
\item An op amp designed to have a low-frequency gain of $10^{5}$ and a high-frequency response dominated by a single pole at 100 rad/s, acquires, through a manufacturing error, a pair of additional poles at 10,000 rad/s. 
\begin{enumerate}
\item At what frequency does the total phase shift reach 180$\degree$ ? 
\item At this frequency, for what value of $\beta$, assumed to be frequency independent, does the loop gain reach a value of unity? 
\item What is the corresponding value of closed-loop gain at low frequencies?
\end{enumerate}
\solution
\begin{align}
G(s) &= \frac{A}{1+\frac{s}{p}} 
\\
A &= \text{Low Frequency Gain} = 10^{5}
\\
p &= 100
\\
G(s) &= \dfrac{10^{5}}{(1+\dfrac{s}{100})(1+\dfrac{s}{10^{4}})^{2}}
\\
\measuredangle G(j\omega) &= -\tan^{-1}\frac{\omega}{100} - 2\tan^{-1}\frac{\omega}{10^{4}}
\end{align}
\item Calculating the frequency at which the total phase shift reach 180$\degree$ 

At $\omega_{180}, \measuredangle G(j\omega_{180}) = -180\degree$

Also $\omega_{180} >> 100$
\begin{align}
180\degree &= 90\degree  + 2 \tan^{-1}(\frac{\omega_{180}}{10^{4}})
\\
\tan^{-1}\frac{\omega_{180}}{10^{4}}&= 45\degree
\\
\frac{\omega_{180}}{10^{4}} &= \tan 45\degree = 1
\\
\omega_{180} &= 10^{4} rad/s
\end{align}
\item Calculating feedback factor $\beta$ for which loop gain at $\omega_{180}$ is unity
\begin{align}
\text{Loop Gain} = G(s)\beta = 1
\\
\dfrac{10^{5}\beta}{\sqrt{1^{2} + (\dfrac{\omega_{180}}{10^{2}})^{2}} \sqrt{(1+\dfrac{\omega_{180}}{10^{4}})^{2}}} = 1
\\
\beta = 0.002
\end{align}
\item Calculating the closed loop gain at low frequency
Let $H(s)$ be the closed loop Transfer Function.
\begin{align}
H(s) &= \dfrac{G(s)}{1+\beta G(s)}
\\
H(s) &= \dfrac{10^{5}}{1+\beta 10^{5}+ \dfrac{s}{100}}
\\
|H(s)| &= \dfrac{10^{5}}{\sqrt{(200)^{2} + (\frac{s}{100})^{2}}}
\end{align}
At low frequencies
\begin{align}
|H(s)| &= 500 V/V
\end{align}

\begin{table}[!ht]
\centering
\input{./tables/ee18btech11001/table1.tex}
\caption{Obtained Parameters}
\label{table:ee18btech11001_params}
\end{table}

The following code performs all the calculations of above equations and returns the parameters 
\begin{lstlisting}
codes/ee18btech11001/code1.py
\end{lstlisting}


\end{enumerate}